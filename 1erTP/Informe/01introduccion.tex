\section{Introducción} \label{intro}

Este trabajo práctico consiste en escuchar pasivamente en una red de área local (LAN). Esto implica capturar paquetes que circulan
en la red. Los paquetes pueden ser \emph{unicast}, \emph{broadcast} o \emph{multicast}. Los primeros son los paquetes dirigidos a una
MAC en particular (dirección de una interfaz de red de algún host en la red), los segundos son los dirigidos a todas las MACs en la red
y los últimos son los dirigidos a un cierto grupo de MACs específico.

Nuestro objetivo será capturar los paquetes del protocolo ARP (Address Resolution Protocol). Este protocolo, según {\tt RFC 826},
está diseñado para que todos los dispositivos en una red puedan encontrar la MAC address de una IP (asumiendo que se utiliza el protocolo
IP a nivel de red). Cada vez que un dispositivo con dirección ({\tt MAC1},{\tt IP1}) quiera enviar un paquete a la {\tt IP2},
si no conoce la dirección física de esta IP, envía un paquete ARP broadcast, preguntando quién tiene la {\tt IP2}. Luego se espera que
el dispositivo con {\tt IP2}, responda un mensaje unicast a {\tt MAC1}, indicando {\tt MAC2}.

A continuación vamos a establecer la notación para los paquetes ARP, utilizada en este informe:
\begin{description}
 \item[MAC\_SRC] Dirección MAC Ethernet del host emisor.
 \item[MAC\_DST] Dirección MAC Ethernet del host receptor (dirección MAC a la cual se envía el paquete, pues el protocolo utiliza nivel de enlace
 para el envío).
 \item[ARP\_OP] Operación del paquete ARP. Puede ser {\tt who-has} ó {\tt is-at}.
 \item[ARP\_MAC\_SRC] Dirección MAC del host emisor indicada en el paquete ARP.
 \item[ARP\_IP\_SRC] Dirección IP del host emisor indicada en el paquete ARP.
 \item[ARP\_MAC\_DST] Dirección MAC del host receptor indicada en el paquete ARP.
 \item[ARP\_IP\_DST] Dirección IP del host receptor indicada en el paquete ARP.
\end{description}

Típicamente MAC\_DST será desconocida y entonces se usará la dirección broadcast ({\tt ff:ff:ff:ff:ff:ff}).

Cada host tendrá una tabla de traducciones de IP-MAC, de tal manera que cuando se desea enviar un paquete a una IP (a nivel de red), si corresponde
a la LAN, se envía utilizando la MAC. Para llenar y actualizar esta tabla es que se utiliza ARP.

El algoritmo utilizado ante la captura de un paquete ARP indicado por {\tt RFC 826} y asumiendo que siempre
se utilizan protocolos Ethernet-IP, es el siguiente:

\begin{enumerate}
 \item En el caso de que ARP\_IP\_SRC esté en nuestra tabla de traducciones, actualizamos la dirección de hardware asociada a ARP\_IP\_SRC. Para esto, reemplazamos el valor actual por ARP\_MAC\_SRC. Caso contrario, verificamos si ARP\_IP\_DST es nuestra IP. En cuyo caso, agregamos (ARP\_IP\_SRC,ARP\_MAC\_SCR) a la tabla de traducciones.
 \item Por otro lado, si ARP\_IP\_DST es nuestra IP y ARP\_OP={\tt who-has}:
  \begin{enumerate}
   \item $swap$(ARP\_IP\_DST, ARP\_IP\_SRC)
   \item $swap$(ARP\_MAC\_DST, ARP\_MAC\_SRC)
   \item $set$(ARP\_MAC\_SRC, nuestra dirección física)
   \item $set$(ARP\_OP, {\tt is-at})
   \item Enviamos el paquete a MAC\_DST = ARP\_MAC\_DST, respondiendo al host que nos haya enviado el paquete, y colocamos MAC\_SRC = ARP\_MAC\_SRC.
  \end{enumerate}
\end{enumerate}

Observar que la primera parte del algoritmo se realiza independientemente de la operación (ARP\_OP). Esto permite que quien reciba un paquete {\tt is-at}
registre la dirección física y no envíe ninguna respuesta. Más aún, el algoritmo es muy laxo en cuanto a la validación de los campos del paquete,
con lo cual podríamos establecer muchas variantes del funcionamiento típico mencionado previamente, que se acoplen a este algoritmo.

El objetivo del trabajo consiste en extraer propiedades características de una red utilizando la información provista por los paquetes ARP que
circulan en la misma.