\documentclass[a4paper,11pt]{article}
\usepackage[paper=a4paper, left=1.5cm, right=1.5cm, bottom=1.5cm, top=1.5cm]{geometry}
\usepackage{ucs}
\usepackage[utf8x]{inputenc}
\usepackage{amsmath}
\usepackage{amsfonts}
\usepackage{amssymb}
\usepackage{amsthm}
\usepackage[latin,spanish]{babel}
\usepackage{fontenc}
\usepackage{graphicx}
%\usepackage{epstopdf}
\usepackage{caratula}

% \usepackage{algpseudocode}
% \usepackage{algorithmicx}
% \usepackage{color}
% \usepackage{pdfpages}
% \usepackage{tikz}
%%%%%% Si se usa tikz para dibujar digrafos, cuando pongas flechas, tira un conflicto. Se soluciona con estas dos lineas:
% \usepackage[matrix,arrow]{xy}
% \usetikzlibrary{intersections}

% \usepackage{dsfont}
% \newcommand{\real}{\mathds{R}}
% \newcommand{\nat}{\mathds{N}}
% \newcommand{\complejos}{\mathds{C}}
% \newcommand{\Subesp}{\mathds{S}}

\usepackage{hyperref}
\hypersetup{%
 % Para que el PDF se abra a pagina completa.
  pdfstartview= {FitH \hypercalcbp{\paperheight-\topmargin-1in-\headheight}},
  pdfauthor={Gaston Requeni},
  pdfsubject={Preinforme - Final Orga2},
 %pdfkeywords={keyword1} {key2} {key3},
 colorlinks=true,
  linkcolor=black,
  urlcolor=blue
}

\author{Gastón Requeni}
\title{TITULO}
\date{DD/MM/AAAA}

\parskip = 4pt


%%%% UNA MAGIA PARA QUE LA BIBLIOGRAFIA SEA UNA SECCION MAS
\makeatletter
\renewenvironment{thebibliography}[1]
     {\section{\bibname}% <-- this line was changed from \chapter* to \section*
      \@mkboth{\MakeUppercase\bibname}{\MakeUppercase\bibname}%
      \list{\@biblabel{\@arabic\c@enumiv}}%
           {\settowidth\labelwidth{\@biblabel{#1}}%
            \leftmargin\labelwidth
            \advance\leftmargin\labelsep
            \@openbib@code
            \usecounter{enumiv}%
            \let\p@enumiv\@empty
            \renewcommand\theenumiv{\@arabic\c@enumiv}}%
      \sloppy
      \clubpenalty4000
      \@clubpenalty \clubpenalty
      \widowpenalty4000%
      \sfcode`\.\@m}
     {\def\@noitemerr
       {\@latex@warning{Empty `thebibliography' environment}}%
      \endlist}
\makeatother

\begin{document}

\input{00caratula.tex}
\newpage

\tableofcontents
\newpage

\section{Introducción} \label{intro}

El nivel de red del modelo OSI introduce el \emph{router} con el objetivo de interconectar redes de nivel 2. 
Para minimizar el delay y evitar ciclos, se utilizan distintos algoritmos, conocidos como algoritmos de routeo (RIP, OSPF, BGP, etc.).
Estos algoritmos generan una red lógica, que establece rutas en la red física para alcanzar una IP pública en alguna parte del mundo desde cada router.
Las rutas no son fijas y podrían variar todas las veces que fuera necesario (por ejemplo si un nodo o enlace deja de funcionar). Uno de los
objetivos de este trabajo es \emph{mostrar que la probabilidad de que una ruta se modifique es muy baja}. También nos interesará analizar cuáles son las
\emph{rutas geográficas} atravesadas para llegar a distintos puntos del mundo.

El Round Trip Time de un enlace punto a punto es fácil de calcular conociendo las propiedades del medio que conecta los dos hosts. Agregar
switches y otros hosts a la conexión complica este cálculo porque los RTTs dependen de las distintas propiedades de los medios interconectados por los
switches. Si además agregamos routers que separan distintas redes switcheadas, el RTT desde un emisor a un receptor es altamente variable
y muy difícil de calcular de manera teórica. Además no sólo dependerán de las propiedades de los medios de transmisión en cada red sino además del
tiempo en que cada paquete permanezca encolado en cada router, que podría variar de router a router. Esto motiva otro de los objetivos del trabajo:
\emph{estimar empíricamente el RTT de cada salto} (i.e. entre un host y un router o entre dos routers).

Finalmente, Internet tal como existe hoy en día está compuesta de varios backbones que llevan la mayor parte del tráfico mundial, que luego
es distribuído por sistemas autónomos de distintos tamaños. Algunas de las conexiones más destacadas son las de fibra óptica submarinas.
Para poder llegar desde un continente a otro en algún momento hay que atravesar el océano mediante estos cables. Por la longitud que tienen,
es esperable que el RTT de esos enlaces sea elevado. Entonces otro objetivo del trabajo será \emph{identificar los enlaces submarinos} basándonos en el RTT
medido y en las rutas geográficas calculadas.
\newpage

\input{02desarrollo.tex}
\newpage

\input{03resultados.tex}
\newpage

\section{Discusión}

\subsection{Red Laboratorios DC}

Vamos a analizar primero los grafos, utilizando los histogramas de ips solicitadas y solicitantes como complemento.

En el primero de los grafos (Figura \ref{grafo-redlabos-1}, superior), observamos la red en general. Es interesante destacar que sólo graficando los
envíos de paquetes ARP {\tt who-has}, el grafo nos muestra una aproximación de la topología de la red. Es difícil mostrar la IP que aparece en cada nodo 
(dado el tamaño del grafo), pero haciendo zoom pudimos observar que todas las IPs correspondientes al círculo y abanico  más grandes del grafo, tienen
la forma {\tt 10.2.100.X} (con la excepción de algunas IPs públicas, que fueron consideradas en los casos atípicos, ver sección \ref{paquetes-destacados}).
Estas IPs corresponden a la parte wifi de la red (esto lo sabemos porque ingresamos a la red desde una laptop y controlamos la IP). Si observamos a la
derecha del grafo, cada uno de los círculos conexos corresponde a un laboratorio distinto (ver Figura \ref{grafo-redlabos-2} para una imágen ampliada).
En cada círculo aparecen IPs del tipo {\tt 10.2.[LABO].X}, donde {\tt X} puede ser una máquina o el/los switch/es.

En la parte wifi de la red, hay dos nodos bien distinguidos: {\tt 10.2.100.254} y {\tt 10.2.100.35}.
\begin{itemize}
 \item Si observamos el zoom realizado en la Figura 
\ref{grafo-redlabos-1}, {\tt 10.2.100.254} recibe y envía muchos paquetes ARP, mientras que {\tt 10.2.100.35} sólo envía. Ésto se ve reflejado en los
histogramas: en IPs Solicitadas (Figura \ref{solicitadas-redlabos}) no aparece {\tt 10.2.100.35}, pero sí {\tt 10.2.100.254}, mientras que en
IPs Solicitantes (Figura \ref{solicitantes-redlabos}) aparecen ambas direcciones (de hecho son las que más veces realizaron solicitudes).
\item Si observamos el grafo de Figura \ref{grafo-redlabos-1}, la {\tt 10.2.100.35} se conecta con casi todos los nodos del subgrafo conexo al que pertenece,
mientras que {\tt 10.2.100.254} se conecta sólo al círculo.
\end{itemize}

Todo esto nos lleva a pensar que {\tt 10.2.100.254} es el router/switchL3 wifi con acceso a internet
y {\tt 10.2.100.35} es otro access point wifi que extiende la red a otro sector físico. Esto explicaría la gran cantidad de solicitudes recibidas
por {\tt 10.2.100.254} (todos buscan el router cuando se registran en la red, celulares, tables, laptops, etc),
pero nadie pide la dirección de {\tt 10.2.100.35} porque es una extensión en principio desconocida
del primero; además ambos envían muchos paquetes ARP porque les llegan paquetes de internet y deben buscar el dispositivo en la red local al cual entregar
el paquete (necesitan la MAC a partir de la IP). De todas formas esta idea sólo quedará en suposiciones porque no controlamos la red y podríamos estar equivocados.

En la Figura \ref{grafo-redlabos-2}, observamos que en cada círculo hay un comportamiento similar al de la red wifi: más de un nodo distinguido, entre ellos
el nodo {\tt 10.2.X.254}. Además si observamos la IPs solicitantes (Figura \ref{solicitantes-redlabos}), en el histograma refinado a 15 minutos, vemos que
aparecen \emph{todas} las direcciones terminadas en {\tt .254} del grafo. Con lo cual creemos que se trata del switch de cada uno de los labos.

Por último, observemos la IP {\tt 10.2.1.230}: Es la IP más solicitada, incluso fue solicitada al menos una vez por minuto 
(según Figura \ref{solicitadas-redlabos}). Sin embargo ni siquiera aprece entre las IPs solicitantes y si observamos el grafo (ver Figura \ref{grafo-redlabos-2}, la ampliación en el recuadro inferior), hay sólo 7 IPs diferentes que le enviaron paquetes ARP, pero le envian muchos cada una. Todas las
IPs que la solicitan corresponden al laboratorio 1, y según nuestra teoría de que las direcciones {\tt .254} son switches, en este caso el switch es el
que más solicita esta IP. Ante esto se nos ocurren tres posibilidades:
\begin{itemize}
 \item El host estuvo caído durante el tiempo de la captura y hubo muchas solicitudes enviadas sin respuesta.
 \item Es un nodo distinguido, tal vez otro switchL3 que tiene acceso a alguna red interna de uso específico.
 \item Es un \emph{host distinguido}, como por ejemplo un servidor de uso interno.
\end{itemize}

Nuevamente cabe aclarar que nuestras suposiciones podrían ser erróneas.

Finalmente analizaremos la información y la entropía. En el caso de $S_{src}$, se observan muchas IPs por encima de la entropía y proporcionando aproximadamente la misma información: esto se debe a que hay muchos símbolos con baja probabilidad, entonces proporcionan mucha información pero aportan
poco al cálculo de la entropía (que usa la probabilidad para ponderar). Entonces sólo unos pocos símbolos proporcionan menos información que la entropía,
y esos corresponden a las IPs más solicitantes. En cambio en la fuente $S_{dst}$ hay una distribución lineal de las probabilidades de cada símbolo, entonces
la entropía es el valor medio de las probabilidades. En este caso, al refinar por intervalos más pequeños, resulta que todas las IPs más solicitadas
proporcionan información por debajo de la entropía.

\subsection{Red Entrepiso}

Vamos a analizar el grafo, junto a los histogramas de ips solicitadas y solicitantes.

Para poder analizar el grafo, vamos a dividirla en 3 subredes (subgrafos conexos).

La primera que analizaremos es la que posee mayor dimensión. En esta, podemos notar que la subred podría dividirse en dos partes. Por un lado, tenemos a las IPs que se conectan directamente con la IP {\tt 10.1.200.254}. Al hacer zoom, podemos notar que esta parte de la subred contiene a la ip {\tt 10.0.0.253} la cual recibe la mayor cantidad de IPs solicitadas. Esta IP se ubica en el extremo más alejado de {\tt 10.1.200.254} y pertenece a una red privada que recibe únicamente solicitudes de la IP {\tt 10.1.200.60}. 
Por otro lado, tenemos a las IPs que tienen la forma {\tt 10.1.100.X}. Como bien vimos anteriormente, estas IPS corresponden a la red WIFI de los laboratorios del departamento de computación. Es por eso que llegamos a la conclusión de que la IP {\tt 10.1.200.254} es un router/switchL3 WIFI con acceso a internet que extiende a la red WIFI de los laboratorios del departamento de computación. 

Las otras dos subredes a analizar son las que se conectan a las ips {\tt 10.1.11.254} y {\tt 10.1.17.254}. Si observamos los gráficos de las IPs solicitadas/solicitantes, podemos ver que sus valores son muy bajos, al igual que las IPs de los Host conectados a ellas. Esto nos lleva a pensar que estas dos subredes son cableadas y que las dos IPs distinguidas son los switch que las conectan ya que son de la forma {\tt X.X.X.254}.

Finalmente analizaremos la información y la entropía. Se observa una situación muy similar a la de red labos dc. En $S_{src}$ la mayor parte de las IPs
aparecen con más información que la entropía dado que son muchos hosts que envían uno o dos paquetes al switch, en cambio los switchs tienen menor
información porque la probabilidad de que envíen ARP es alta, debido a que solicitan MACs de los hosts conectados. En $S_{dst}$ se observa un
comportamiento similar a $S_{src}$ en el caso particular de esta red.


\subsection{Red Centro de Estudiantes}

El tamaño de esta red facilita el análisis, dado que desde el grafo de relaciones ARP ya se observan dos nodos claramente distinguidos:
{\tt 192.168.1.1} y {\tt 192.168.1.135}. El histograma de IPs más solicitantes confirma esto. 
Además según el histograma de IPs más solicitadas, aparece la IP distinguida {\tt 169.254.255.255}, pero no aparece {\tt 192.168.1.135}, con lo cual
suponemos que este será un access point no principal (situación similar red laboratorios dc).

Finalmente analizaremos la información y la entropía. Se observa una situación muy similar a la de red labos dc. Las IPs correspondientes a los
switchesL3/routers, se corresponden con las IPs que dan menos información que la entropía (en $S_{src}$), y en $S_{dst}$, en el caso de {\tt 192.168.1.1},
aparece con el mismo valor que la entropía.

\newpage

\section{Conclusiones}

Capturando paquetes ARP podemos estimar la topología de la red de área local. Además teniendo en cuenta la cantidad de envíos y la información
de las fuente $S_{src}$ podemos estimar qué IPs corresponden a switches (las que se encuentran por debajo o apenas por encima de la entropía). Esto sucede
porque hay muchas IPs que en la fuente $S_{src}$ aparecen con muy baja probabilidad (muchas IPs que solicitan MACs de otros hosts, por ejemplo el router, pero lo hace una o dos veces y muy esporádicamente),
entonces al calcular la entropía, todas esas quedan por encima y por debajo quedan las IPs que realizan muchas solicitudes (a todos esos mismos hosts mencionados).
Por último, teniendo en cuenta la cantidad de recepciones y la información de la fuente $S_{dst}$ podemos destacar otras IPs que son muy solicitadas
y proveen poca información.

\newpage


\begin{thebibliography}{X}
\bibitem{geolocation} \url{http://plotip.com/}, \url{http://www.geoiptool.com/es/}, \url{http://freegeoip.net/}.
\bibitem{googlemaps} \url{https://maps.google.com.ar}
\bibitem{internetprotocolv4} RFC 791: Internet Protocol (version 4).
\bibitem{peterson} Peterson, Davie. \emph{Computer Networks: A Systems Approach}. 3ra Edición. Sección 6.5.3 (Congestión - Calidad de Servicio - Servicios Diferenciados).
% \bibitem{man-arping} {\tt man arping}
% \bibitem{ettercap} \url{http://ettercap.github.io/ettercap/}, \url{http://linux.die.net/man/8/ettercap} (ver opción {\tt arp} del programa).

\end{thebibliography}

\end{document}