\section{Desarrollo}

Se realizará un análisis sobre 3 redes de área local distintas, intentando identificar características de las mismas en los paquetes del protocolo ARP. 

Las redes elegidas fueron:
\begin{itemize}
\item Red Laboratorios DC: Esta red se sitúa en los laboratorios del departamento de computacion de la facultad de Ciencias Exactas y Naturales. Para poder acceder a la misma, tuvimos que loguearnos en una de las computadoras fijas del laboratorio. Se puede acceder también mediante una conexión
wireless.

\item Red Entrepiso: Esta red se encuentra en los pasillos de la Facultad de Ciencias Exactas y Naturales y en algunas oficinas. Se puede acceder
por Ethernet o Wi-Fi, pero en este caso, accedimos por Wi-Fi para realizar las mediciones.

\item Red Centro de Estudiantes (CECEN): Esta red se encuentra en el pabellón 2 de Ciudad Universitaria, en los alrededores del kiosko del centro
de estudiantes de Exactas. Sólo se puede acceder de manera wireless.

\end{itemize}

El análisis comenzará tomando una muestra de paquetes capturados en cada una de las redes durante {\bf 30 minutos}. Luego usaremos una herramienta
de software para leer esos datos y procesarlos de distintas maneras. En la secciones posteriores, desarrollaremos distintos métodos de análisis
de los datos, y luego mostraremos y discutiremos los resultados.

\subsection{IPs Más Solicitadas}

De entre todos los paquetes de la muestra vamos a quedarnos únicamente con los que sean del tipo {\tt who-has} y broadcast. El objetivo
será buscar las IPs que más fueron solicitadas como destino del {\tt who-has} (ARP\_IP\_DST), es decir las IPs por las que más se preguntó en la red.

Para esto vamos a contar la cantidad de veces que cada IP fue solicitada. Esta metodología tiene el inconveniente de que aparezca una IP con una
cantidad de solicitudes inmensa en un intervalo de tiempo muy reducido. Eso corresponde a un sesgo en la medición, dado que consideramos que una
IP es muy solicitada cuando es solicitada uniformemente a lo largo de los 30 minutos de la captura.

Para evitar en gran medida el sesgo, vamos a tomar intervalos de tiempo de tamaño $n$ minutos ($n$ divisor de 30). Luego vamos a tener en cuenta
sólo a las IPs que hayan sido solicitadas en \emph{todos} los intervalos al menos una vez, y contar las veces que se solicitaron éstas
en los 30 minutos. La elección de $n$ no debería ser arbitraria, es por eso que realizaremos algunos experimentos para determinar el valor de $n$
que muestre los datos más apropiados, y mostraremos en este informe sólo algunos de ellos (los más significativos).

Este análisis se complementa con los grafos de relaciones ARP.

\subsection{IPs Más Solicitantes}

Aquí plantearemos un análisis idéntico al anterior, pero considerando ahora la cantidad de solicitudes realizadas por cada IP, es decir las IPs
que más aparecen como fuente del {\tt who-has} (ARP\_IP\_SRC).

Este análisis también se complementa con los grafos de relaciones ARP.

\subsection{Grafos de relaciones ARP}

Crearemos grafos cuyos nodos se corresponderán con IPs y cuyas aristas con paquetes ARP enviados:
$$ IP1 \longrightarrow_n IP2$$
significa que la $IP1$ envió $n$ veces un {\tt who-has} broadcast cuya ip destino fue $IP2$.

Este grafo permitirá distinguir la cantidad de nodos distintos que referencian a cada IP (sacando las repeticiones que se contabilizaban en IPs solicitadas
y solicitantes). Además nos proporcionará una idea global de las relaciones en la red, permitiendo distinguir uno o más nodos especiales.

\subsection{Información y Entropía}

Para el siguiente análisis consideraremos dos fuentes teóricas:
\begin{itemize}
 \item $S_{src} = {s_1, s_2, \dots, s_n}$: Cada símbolo $s_i$ es una dirección IP que aparece en el campo ARP\_IP\_SRC de los paquetes ARP.
 \item $S_{dst} = {d_1, d_2, \dots, d_n}$: Cada símbolo $d_i$ es una dirección IP que aparece en el campo ARP\_IP\_DST de los paquetes ARP. 
\end{itemize}
Por cada paquete ARP capturado, se considera que estas fuentes emitieron un símbolo (ARP\_IP\_SRC ó ARP\_IP\_DST respectivamente).

La idea es analizar la información de cada símbolo y compararla con la entropía de la fuente.

Aquí también utilizaremos la idea de dividir los 30 minutos en intervalos más chicos y calcular la intersección.

\subsection{Paquetes destacados (``raros'')} \label{paquetes-destacados}
En las secciones posteriores estudiaremos algunos paquetes observados que resultaron extraños comparados con el uso estándar del protocolo ARP.

\subsubsection{ARP\_OP=who-has, broadcast, ARP\_IP\_SRC=0.0.0.0}
Encontramos diversos paquetes del estilo:

\begin{center}
\scriptsize
 \begin{tabular}{ | c | c | c | c | c | c | c |}
\hline
 MAC\_SRC & MAC\_DST & ARP\_OP & ARP\_MAC\_SRC & ARP\_IP\_SRC & ARP\_MAC\_DST & ARP\_IP\_DST \\
 \hline
 {\color{red} {\tt 00:27:0e:0d:f3:4b}} & {\color{blue} {\tt ff:ff:ff:ff:ff:ff}} & {\color{blue} {\tt who-has}} & {\tt 00:27:0e:0d:f3:4b} & {\color{blue} {\tt 0.0.0.0}} & {\tt 00:00:00:00:00:00} & {\color{red} {\tt 10.2.5.14}}\\
 \hline
 \multicolumn{7}{|c|}{¿Quién tiene la IP {\tt 10.2.5.14}? Informar a {\tt 0.0.0.0} ({\tt 00:27:0e:0d:f3:4b})}\\
 \hline
\end{tabular}
\end{center}

El paquete fue capturado en ``Red Labos DC''.
Es un paquete broadcast preguntando por una IP en particular, pero la IP fuente es nula. Buscamos otros paquetes correspondientes a la IP
{\tt 10.2.5.14}, y encontramos el siguiente:

\begin{center}
\scriptsize
 \begin{tabular}{ | c | c | c | c | c | c | c |}
\hline
 MAC\_SRC & MAC\_DST & ARP\_OP & ARP\_MAC\_SRC & ARP\_IP\_SRC & ARP\_MAC\_DST & ARP\_IP\_DST \\
 \hline
 {\color{red} {\tt 00:27:0e:0d:f3:4b}} &{\tt ff:ff:ff:ff:ff:ff} &{\tt who-has} &{\tt 00:27:0e:0d:f3:4b} &{\color{red} {\tt 10.2.5.14}} &{\tt 00:00:00:00:00:00} &{\tt 10.2.5.249}\\
 \hline
 \multicolumn{7}{|c|}{¿Quién tiene la IP {\tt 10.2.5.249}? Informar a {\tt 10.2.5.14} ({\tt 00:27:0e:0d:f3:4b})}\\
 \hline
\end{tabular}
\end{center}

Esto nos llamó mucho la atención, y luego de realizar
una investigación al respecto, encontramos que estos paquetes se usan para detectar direcciones IP duplicadas\cite{linux-ip}\cite{man-arping}.
Como se observa en el ejemplo,
la IP {\tt 10.2.5.249} corresponde a la MAC address {\tt 00:27:0e:0d:f3:4b}, que es la misma que envió el paquete extraño que mencionamos primero.

Si observamos el algoritmo de un host receptor de un paquete ARP (sección \ref{intro}): Si el host receptor tiene la misma IP que el emisor,
en el paso 2 responderá con un paquete que indique su MAC. Entonces el host emisor, al recibirlo, detectará que hay una IP duplicada.

\subsubsection{ARP\_OP=who-has, broadcast, ARP\_IP\_SRC=ARP\_IP\_DST (ARP gratuito)}
Es una técnica utilizada para \emph{anunciar} que un host es \emph{dueño} de una IP. Los demás hosts de la red, cuando reciban este mensaje,
pueden tenerlo en cuenta o no: Si alguien tiene la ip duplicada, puede modificarla para evitar que haya duplicados o simplemente ignorar el
mensaje y seguir teniendo la ip duplicada, o tal vez enviar otro ARP gratuito (dar pelea...). En UNIX se respeta y acepta la ip de un host que
envía un ARP gratuito. Ver \cite{linux-ip}.

Los paquetes tienen este formato: (ejemplo extraído de Red Labos DC)
\begin{center}
\scriptsize
 \begin{tabular}{ | c | c | c | c | c | c | c |}
\hline
 MAC\_SRC & MAC\_DST & ARP\_OP & ARP\_MAC\_SRC & ARP\_IP\_SRC & ARP\_MAC\_DST & ARP\_IP\_DST \\
 \hline
 {\tt b8:af:67:a1:ea:9e} & {\color{blue} {\tt ff:ff:ff:ff:ff:ff}} & {\color{blue} {\tt who-has}} & {\tt b8:af:67:a1:ea:9e} & {\color{blue} {\tt 10.2.0.187}} & {\tt 00:00:00:00:00:00} & {\color{blue} {\tt 10.2.0.187}}\\
 \hline
 \multicolumn{7}{|c|}{{\tt b8:af:67:a1:ea:9e} anuncia que tiene la IP {\tt 10.2.0.187}}\\
 \hline
\end{tabular}
\end{center}

Este formato también sirve para hacer ARP spoofing. Un host envía este paquete y, en una red de host UNIX, podría empezar a recibir los paquetes
destinados a otra IP. A partir de eso, se podrían modificar los paquetes y reenviarlos al verdadero destinatario, o simplemente hacerce pasar
por el destinatario y responder al emisor. Para más detalles sobre este uso, ver \cite{ettercap}.

Existe otra manera de generar un ARP gratuito, que es enviando un paquete {\tt is-at} broadcast, donde ARP\_MAC\_SRC = ARP\_MAC\_DST, ARP\_IP\_SRC = ARP\_IP\_DST (ver \cite{linux-ip}). No observamos paquetes de este estilo en nuestras mediciones.

\subsubsection{ARP\_OP=who-has, unicast}

Los switchs a veces envían paquetes {\tt who-has} unicast para confirmar que un host sigue levantado. Este procedimiento se conoce típicamente como ARPING
\cite{man-arping}.

\subsubsection{ARP\_OP=is-at, broadcast, no es ARP gratuito}

No logramos encontrar ninguna explicación a esto.

\subsubsection{ARP\_OP=who-has, broadcast, ARP\_IP\_DST = ip pública}

No logramos encontrar ninguna explicación a esto.