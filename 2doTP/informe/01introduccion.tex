\section{Introducción} \label{intro}

El nivel de red del modelo OSI introduce el \emph{router} con el objetivo de interconectar redes de nivel 2. 
Para minimizar el delay y evitar ciclos, se utilizan distintos algoritmos, conocidos como algoritmos de routeo (RIP, OSPF, BGP, etc.).
Estos algoritmos generan una red lógica, que establece rutas en la red física para alcanzar una IP pública en alguna parte del mundo desde cada router.
Las rutas no son fijas y podrían variar todas las veces que fuera necesario (por ejemplo si un nodo o enlace deja de funcionar). Uno de los
objetivos de este trabajo es \emph{mostrar que la probabilidad de que una ruta se modifique es muy baja}. También nos interesará analizar cuáles son las
\emph{rutas geográficas} atravesadas para llegar a distintos puntos del mundo.

El Round Trip Time de un enlace punto a punto es fácil de calcular conociendo las propiedades del medio que conecta los dos hosts. Agregar
switches y otros hosts a la conexión complica este cálculo porque los RTTs dependen de las distintas propiedades de los medios interconectados por los
switches. Si además agregamos routers que separan distintas redes switcheadas, el RTT desde un emisor a un receptor es altamente variable
y muy difícil de calcular de manera teórica. Además no sólo dependerán de las propiedades de los medios de transmisión en cada red sino además del
tiempo en que cada paquete permanezca encolado en cada router, que podría variar de router a router. Esto motiva otro de los objetivos del trabajo:
\emph{estimar empíricamente el RTT de cada salto} (i.e. entre un host y un router o entre dos routers).

Finalmente, Internet tal como existe hoy en día está compuesta de varios backbones que llevan la mayor parte del tráfico mundial, que luego
es distribuído por sistemas autónomos de distintos tamaños. Algunas de las conexiones más destacadas son las de fibra óptica submarinas.
Para poder llegar desde un continente a otro en algún momento hay que atravesar el océano mediante estos cables. Por la longitud que tienen,
es esperable que el RTT de esos enlaces sea elevado. Entonces otro objetivo del trabajo será \emph{identificar los enlaces submarinos} basándonos en el RTT
medido y en las rutas geográficas calculadas.